% Copyright 2015 by Özhan Kaya (Fizzzo42@gmail.com)

% Telling sharelatex which compiler has to be used
% !TEX program = pdflatex

\documentclass[11pt,a4paper,sans]{moderncv}   % ('10pt', '11pt' and '12pt'), ('a4paper', 'letterpaper', 'a5paper', 'legalpaper', 'executivepaper' or 'landscape') , ('sans' or 'roman')

% ModernCV Themes
\moderncvstyle{banking}                        % 'casual' (default),'classic', 'oldstyle' or 'banking'
\moderncvcolor{orange}                          % 'blue', 'orange', 'green', 'red', 'purple', 'grey' or 'black'
%\nopagenumbers{}

% ajustes para los margenes de pagina
\usepackage[scale=0.75]{geometry}
%\setlength{\hintscolumnwidth}{3cm}

\usepackage[utf8]{inputenc}					% Added so I can use special letters like an "Ö"


% Personal data
\name{Özhan}{Kaya}
\title{CV}
\address{Ufmattenstrasse 57, 8303 Bassersdorf}
\phone[mobile]{+41792840813}
\phone[fixed]{+413641176}
%\phone[fax]{+3~(456)~789~012}
\email{Fizzzo42@gmail.com}
\homepage{www.kayaapps.ch}
%\extrainfo{informacion adicional}
\photo[64pt][0.4pt]{picture}
\quote{Up to date version of this CV is available \href{https://github.com/Fizzzo42/cv}{\textbf{here}}}
\social[github]{Fizzzo42}

% New command "makecvtitle" (needed to add image on top of title)
\makeatletter
\@ifpackageloaded{moderncvstylebanking}{%
\let\oldmakecvtitle\makecvtitle
\renewcommand*{\makecvtitle}{%
  {\centering\framebox{\includegraphics[width=\@photowidth]{\@photo}}\par\vspace{10pt}}%
  \oldmakecvtitle%
}%
}{%
}
\makeatother


\begin{document}
%\begin{CJK*}{UTF8}{gbsn}                     % para redactar el CV en chino usando CJK
%\maketitle
\makecvtitle

\section{Education}
\cventry{2013--2016}{HSR Hochschule für Technik Rapperswil}{Bachelor of Science FHO in Computer Science}{Rapperswil}{\textit{Note: -}}{Vollzeit}
\cventry{2012--2013}{BBW Berufsbildungsschule Winterthur}{Berufsmaturität (Technische Richtung)}{Winterthur}{\textit{Note: -}}{Vollzeit}
\cventry{2008--2012}{MSW Mechatronik Schule Winterthur}{Berufslehre Elektroniker Profil E}{Winterthur}{\textit{Note: 5.1}}{EFZ, IPA Note: 5.8}

\section{Tesis de maestr\'ia}
\cvitem{t\'itulo}{\emph{T\'itulo}}
\cvitem{sinodares}{Sinodales}
\cvitem{descripci\'on}{Una breve descripci\'on de la tesis}

\section{Experiencia}
\subsection{Vocacional}
\cventry{a\~no--a\~no}{t\'itulo del puesto}{Empleador}{Ciudad}{}{Descripci\'on general, no m\'as de 1 \'o 2 l\'ineas.\newline{}%
Detalle de logros:%
\begin{itemize}%
\item Logro 1;
\item Logro 2, con sub-logros:
  \begin{itemize}%
  \item Sub-logro (a);
  \item Sub-logro (b), con sub-sub-logros (¡evite hacer esto!);
    \begin{itemize}
    \item Sub-sub-logro i;
    \item Sub-sub-logro ii;
    \item Sub-sub-logro iii;
    \end{itemize}
  \item Sub-logro (c);
  \end{itemize}
\item Logro 3.
\end{itemize}}
\cventry{a\~no--a\~no}{t\'itulo del puesto}{Empleador}{Ciudad}{}{Descripci\'on l\'inea 1\newline{}Descripci\'on l\'inea 2}
\subsection{Miscel\'aneo}
\cventry{a\~no--a\~no}{t\'itulo del puesto}{Empleador}{Ciudad}{}{Descripci\'on}

\section{Idiomas}
\cvitemwithcomment{Idioma 1}{nivel}{Comentario}
\cvitemwithcomment{Idioma 2}{nivel}{Comentario}
\cvitemwithcomment{Idioma 3}{nivel}{Comentario}

\section{Conocimientos de computaci\'on}
\cvdoubleitem{categor\'ia 1}{XXX, YYY, ZZZ}{categor\'ia 4}{XXX, YYY, ZZZ}
\cvdoubleitem{categor\'ia 2}{XXX, YYY, ZZZ}{categor\'ia 5}{XXX, YYY, ZZZ}
\cvdoubleitem{categor\'ia 3}{XXX, YYY, ZZZ}{categor\'ia 6}{XXX, YYY, ZZZ}

\section{Interests}
\cvitem{hobby 1}{Descripci\'on}
\cvitem{hobby 2}{Descripci\'on}
\cvitem{hobby 3}{Descripci\'on}

\section{Extra 1}
\cvlistitem{Tema 1}
\cvlistitem{Tema 2}
\cvlistitem{Tema 3}

\renewcommand{\listitemsymbol}{-~}            % para cambiar el simbolo para las listas

\section{Extra 2}
\cvlistdoubleitem{Tema 1}{Tema 4}
\cvlistdoubleitem{Tema 2}{Tema 5\cite{book1}}
\cvlistdoubleitem{Tema 3}{}

% Las publicaciones tomadas de un archivo de BibTeX sin usar multibib\renewcommand*{\bibliographyitemlabel}{\@biblabel{\arabic{enumiv}}}

\nocite{*}
\bibliographystyle{plain}
\bibliography{publications}                   % 'publications' es el nombre del archivo BibTeX

% Las publicaciones tomadas de un archivo BibTeX usando el paquete multibib
%\section{Publicaciones}
%\nocitebook{book1,book2}
%\bibliographystylebook{plain}
%\bibliographybook{publications}              % 'publications' es el nombre del archivo BibTeX
%\nocitemisc{misc1,misc2,misc3}
%\bibliographystylemisc{plain}
%\bibliographymisc{publications}              % 'publications' es el nombre del archivo BibTeX

%\clearpage\end{CJK*}                          % si esta redactando su CV en chino usando CJK, \clearpage es requerido por fancyhdr para que funcione correctamente con CJK, aunque esto eliminara la numeracion de pagina al dejar \lastpage como no definido
\end{document}


%% fin del archivo `template-es.tex'.

